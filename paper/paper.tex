% !TeX spellcheck = en_GB
% This file provides a template to format JASSS articles v.0.5, 2015-03-24

% In order to be compiled, the following packages should be installed in your system:
% graphicx,xcolor, booktabs,amsmath, ifthen, geometry, authblk, natbib, endnotes
 
 % Please use pdflatex

% The font used is Source Sans Pro, normally included in Tex Live and other  major LaTeX distributions
% Location at CTAN: http://www.ctan.org/tex-archive/fonts/sourcesanspro/
% See also: http://www.tug.dk/FontCatalogue/sourcesanspro/

%%%%%%%%%%%%%%%%%%%%%%%%%%%%%%%%%%%%%%%%%%%%%%
\documentclass{JASSS}
\usepackage[utf8]{inputenc}
%%%%%%%%%%%%%%%%%%%%%%%%%%%%%%%%%%%%%%%%%%%%%%

% Editorial fields (to be set in case of publication)	
% Please leave this section untouched
%\doinum{10.18564/jasss.xxxx}
%\volume{xx}
%\issue{x}
%\article{x}
%\pubyear{20xx}
%\received{dd-mmm-yyyy}
%\accepted{dd-mmm-yyyy}
%\published{dd-mmm-yyyy}

%%%%%%%%%%%%%%%%%%%%%%%%%%%%%%%%%%%%%%%%%%%%%% 

% title, authors and affiliations	

\title{Modelling Mass Migration based on Spatial\\ Generative Agent-Based Trade Simulation}

%All authors should be included in the submission. To anonymise the submission, set to 'true' the \reviewcopy command below. However, before submitting  you can check that all the informations you entered are correct by temporarily setting it to 'false'. Please remember to set it back to 'true' before the submission.
\reviewcopy{true} 

\author[1]{Manuel Bröchin}
\affil[1]{Swiss Federal Institute of Technology in Zurich, Rämistr. 101, CH-8092 Zürich, Switzerland}

% Subsequent author should be included using the following template. You can add more in case of need, just remember to appropriately set the corresponding number.  Please check the the authblk package documentation  in case of doubts

\author[1]{Oliver Blaser}

\author[1]{Renato Menta}

%\author[4]{fourth author here}
%\affil[4]{fourth of the third author here}

% In case of multiple affiliation for the same author:

%\author[1,2]{Author name here}
%\affil[1]{First affiliation}
%\affil[2]{Second affiliation}

%  In case of several authors sharing the same affiliation:

%\author[1]{First author name}
%\author[1]{Second author name}
%\affil[1]{Affiliation}

% email for the corresponding author
\email{rmenta@ethz.ch}

%%%%%%%%%%%%%%%%%%%%%%%%%%%%%%%%%%%%%%%%%%%%%%

% NOTES. Please use endnotes. Notes should be placed after the main text and appendices and before the references. Also remember to uncomment the \theendnotes commands at the end of the document (just before the references)

%%%%%%%%%%%%%%%%%%%%%%%%%%%%%%%%%%%%%%%%%%%%%%

% REFERENCES should be included using the \citep, \citet, etc. commands provided by the natbib package

\usepackage{natbib}
	\setcitestyle{authoryear,round,aysep={}}
	
%%%%%%%%%%%%%%%%%%%%%%%%%%%%%%%%%%%%%%%%%%%%%%	

% EXTRA PACKAGES. Please place here any extra package you need along with your own command definitions
\DeclareMathOperator*{\argmin}{arg\,min}
\DeclareMathOperator*{\argmax}{arg\,max}
\newcommand{\gdf}{\mathit{GDF}}
\newcommand{\wtf}{\mathit{WTF}}
\newcommand{\clf}{\mathit{CLF}}
\newcommand{\col}{\mathit{COL}}
\newcommand{\mgf}{\mathit{MGF}}

%%%%%%%%%%%%%%%%%%%%%%%%%%%%%%%%%%%%%%%%%%%%%%

\begin{document}
\maketitle 

%%%%%%%%%%%%%%%%%%%%%%%%%%%%%%%%%%%%%%%%%%%%%%

% Abstract and keywords

\begin{abstract}
Our paper presents a new way of modelling mass migration based on a generative-agent-based trade simulation. We abstract the behaviour of social groups to a simple trading pattern following a few basic rules with which we can successfully simulate migration of populations to areas which are more beneficial to them. We investigated the influence of different factors, such as resource shortage and excess, as well as trading patterns which rely on social connections only, contrasting to individual economic interests, ignoring all social constructs. The emerging migration patterns based on the collective behaviour give us an explanation on how and why people in ancient times migrated, as well as on how we might choose our living places in the future. We show that mass migration can be modelled with comparatively simple rules and that we can better understand the tendencies which influenced people in the past, and prepare for future scenarios.
\end{abstract}

\begin{keywords}
Mass Migration, Trade Simulation, Generative Agent Based Model, Collective Behaviour
\end{keywords}

%%%%%%%%%%%%%%%%%%%%%%%%%%%%%%%%%%%%%%%%%%%%%%
% Start of  paragraph numbering. Please leave this untouched
\parano{}

%%%%%%%%%%%%%%%%%%%%%%%%%%%%%%%%%%%%%%%%%%%%%%

% MAIN TEXT

% Place the main text here. Please use only \section, \subsection, and \subsubsection sectioning commands to structure your text. Do NOT use lower sectioning commands, including \paragraph and \subparagraph

%For bulleted, numbered and description lists the class provides three asterisked environments to replace the standard LaTeX ones: itemize*, enumerate*, and description*. Please use these ones as the standard environment may cause issues with the paragraph numbering system.

%\begin{itemize*}
%     \item 
%     \item
%     \item
% \end{itemize*} 

% hyperlinks (to models, videos, etc.) can be included via the \href command (remember to put \usepackage{hyperref} in the preamble). Check the hyperref documentation for details

%%%%%%%%%%%%%%%%%%%%%%%%%%%%%%%%%%%%%%%%%%%%%%

\section{Introduction}
Increasing migration pressures our society into adapting quickly and finding appropriate solutions to deal with those sudden changes. The development of communication and social networks contribute to information being shared quickly over large areas, resulting in more people being able to compare their situation to others and trying to balance out those differences by moving to areas with higher benefits. The scientific work concerning migration has been dominated by the gravity model, which is, unlike our simulation, based on observation and is thus biased. The spatial interaction, on which our simulation is built, has been modelled with different approaches than the gravity model, though none of which used trade transactions as working base.

This new approach allows us to compare different migration scenarios based solely on trading. This gives a rough, yet information rich simulation with which we can anticipate the general behaviour of big populations in critical scenarios. These scenarios include, but are not limited to shortage of vital resources such as water or escape from political persecution. It depends on how the resources used in our model are interpreted.

We use a generative agent-based model, in which the agents, connected by a network, are placed on a spatial grid on which they can move around according to their rules. Trades are one-sided and consist of one agent giving resources to another, which will make the receiving agent move towards the other one. The grid is divided into multiple parcels, on which the resources are distributed according to a distribution function. To further describe our model, we use the ODD protocol \citep{odd}.

\section{Model}
\subsection{Purpose}
The start of our investigation is the question of how much influence social behaviour and social structure have on the success of a civilization of humans. We chose to use the action of trading as condensate of all action. This is of course an under-approximation of all actions a human takes. Real humans have a multitude of motives such as health, love, happiness and many more. Nevertheless we follow the tradition of assuming a homo economicus who always tries to maximize utility or profit \citep{rittenberg}. As social behaviour we define the preferences someone has towards his peers: if he has the choice between two individuals as trading partners, which one does he prefer and what factors help him make his decision? Is he utalitarian and as such tries to maximize the utility of his actions? Or is he egoist and tries to maximize his own benefit? We model the social structure twofold: humans have logical connections corresponding to relations and physical connections in the form of their position on the map. We will explain further why we think that this is a useful approximation of a human civilization in the section about Design Concepts. With these abstractions our model allows testing of what kind of social behaviour and structure is most successful.


\subsection{Entities, state variables, and scales}
Our model uses homogenous agents. Each agent represents a human being. Agents act according to two sets of variables: Private values, which are different for each agent and can change over time and influence the behaviour of a specific agent, and static values, which are the same for each agent and determine the coarse behaviour of the agents.\\

\textbf{\underline{Private variables}}:
\begin{description*}
	\item[Resources:]
		Variable ranging from 0 to (potentially) infinity. This represents how much an agent possesses and influences his trading behaviour.

	\item[Sympathy:]
		Variable which represents how much an agent likes one specific neighbour. It influences its trading behaviour and is adjusted as explained below.

	\item[Position on Grid:]
		The $x$- and $y$-coordinates of an agent on the plane. This only represents the spatial location, disregarding its position in the network.
\end{description*}

\textbf{\underline{Static variables}}:
\begin{description*}
	\item[willingness-to-trade-factor ($\wtf$):]
		Percentage representing the chance that an agent actually makes a trade with someone else, independently of all the other factors.

	\item[gold-dig-factor ($\gdf$):]
		Ratio representing how an agent decides on his trading partner. It can be visualized with the following fraction:
		\begin{equation*}
			\gdf = \frac{\text{\textit{Sympathy based trading}}}{\text{\textit{Economics based trading}}}
		\end{equation*}
		If the value is close to 0, the agent will choose a partner which he likes, disregarding the amount of resources this potential partner has. If the value is close to 1, the agent will choose a partner which has lots of resources, disregarding how much he likes him.

	\item[circle-of-life-factor ($\clf$):]
		Percentage representing the chance that an agent dies or spawns another agent (representing sexual reproduction) in any given iteration independent of other factors or variables.
		
	\item[Cost of Life ($\col$):]
		The amount of resources which an agent has to spent in each tick. This value is also used to determine the distribution of resources.
\end{description*}

The plane is divided into a parcel-grid. In each iteration (tick), every parcel gets resources allocated according to a distribution function (see below). After the resources are distributed, all agents on a specific parcel will share the resources on it. Resources on empty parcels are not used and get overwritten in the next iteration. The partitioning into parcels allows us to simulate starvation caused by overpopulation (similar to the real world): since all agents located on a specific parcel share it's resources, it's counter-beneficial for too many agents to be on the same parcel.\\

To determine how the resources are distributed, we use a variable called \textbf{merciful-god-factor} ($\mgf$). If the $\mgf$ is equal to 1, the sum of all resources distributed on the plane are sufficient for all agents to survive. If the $\mgf$ is above 1, there is an excess of resources, if the $\mgf$ is below 1, there is a shortage (linearly dependent on the $\mgf$).

\subsection{Process overview and scheduling}
Our simulation works with synchronous updates (ticks), meaning all agents do the same things at the same time. The actions in each tick can be divided into four main parts:

\subsubsection{Resource distribution}
	To distribute the resources on the grid, we us the function
	\begin{equation}
		f_\mathit{dist}(\mathit{pos},\, \mathit{time}) \to [0,\, 1]
	\end{equation}
	which takes the position of a parcel as an input and calculates the percentage which determines how much of the total resources this specific parcel receives. To obtain the actual amount of resources distributed, we multiply this percentage with the total need for resources and the $\mgf$:
	\begin{equation}
		resource = f_\mathit{dist}(\mathit{pos},\, \mathit{time}) \cdot n \cdot \col \cdot \mgf
	\end{equation}
	where $n$ is the initial number of agents. Using this formula, we see that the total distribution of resources is equal to the total needs:
	\begin{alignat}{2}
		&\sum_{p \in \mathit{Parcels}} f_\mathit{dist}(\mathit{p.pos},\, \mathit{time}) \cdot n \cdot \col &&= n \cdot \col\\
		\Longleftrightarrow \quad &\sum_{p \in \mathit{Parcels}} f_\mathit{dist}(\mathit{p.pos},\, \mathit{time}) &&= 1
	\end{alignat}
	
\subsubsection{Trade requests}
	Now all agents have the possibility to propose a trade with exactly one of their neighbours. They do this by finding the best match for a trading partner in the list of their neighbours. The best trading partner is found by evaluating the following formula:
	\begin{equation}\label{eq: trade_req}
		\argmax_{a \in \mathit{Neighbours}} \left( \gdf \cdot a.\mathit{resource} + (1-\gdf) \cdot \mathit{sympathy}(a) \right)
	\end{equation}
	where $sympathy()$ is the mapping from a neighbour of an agent to a sympathy value.
	
	
\subsubsection{Trade transactions}
	 Now all agents can view the list of their neighbours that have proposed a trade with them. The individual willingness-to-trade factor decides with what probability an agent considers accepting a trade in each tick. If an agent decides to accept a trade he finds a best match for a trading partner with the following formula (note the similarities to equation \ref{eq: trade_req}):
	 \begin{equation}
		 \argmin_{a \in \mathit{Neighbours}} \left( \gdf \cdot a.\mathit{resource} - (1-\gdf) \cdot \mathit{sympathy}(a) \right)
	 \end{equation}
	 A trade causes the sympathy of the giving agent towards the receiving agent to sink by 1, and conversely for the receiving agent towards the giving agent to rise by 2. This mechanism ensures that frequent interactions between a pair of agents results in a strengthened relationship:
	 \begin{equation}
	 	\lim_{n \to \infty} n \cdot (p\cdot 2 - (1 - p)\cdot 1) = \lim_{n \to \infty} n \cdot (3\cdot p - 1) = \begin{cases}
	 	= \infty,\quad &\text{if $p > \frac{1}{3}$}\\
	 		= 0, &\text{if $p = \frac{1}{3}$}\\
	 		= -\infty, &\text{if $p < \frac{1}{3}$}
	 	\end{cases}
	 \end{equation}
	 where $p$ denotes the ratio of mutual trading. This means, that if agent $A$ does not trade more than twice as often with agent $B$ than the other way around, their respective sympathy will increase linearly over time.

\subsubsection{Agent movements}
		After all the trades have been completed the agents will update their position according to the following schema:
\begin{enumerate*}
	\item
		The agent checks whether he has received more resources this tick than in the previous one, if so, he keeps his position, else he moves to his previous position.

	\item
		If the agent has received resources from another agent, he will move towards his trading partner.

	\item
		The agent will move in a random direction (this movement is smaller compared to first type of movement). 
\end{enumerate*}
The first and second movement together ensure collective learning in such that agents follow the leads of other agents to more prosperous regions of the map. The random movement helps avoiding local minima. This concludes one tick and the same process starts over again.


\subsection{Design concepts}

\subsubsection{Basic Principles}
	In this section we will explain some of our design choices. The composition of these choices is what makes our model. First we will examine the individual decisions and explain the thinking behind them and after that we will explain why the composition of them is a useful approximation of the civilization we want to model. As mentioned in the introduction we will again distinguish and investigate along the axes of social behaviour and structure.
	
	\textbf{\underline{Social behaviour}}:
	\begin{description*}
		\item[$\boldsymbol{\gdf}$:]
			The biggest part of our agent's characters is decided by the value of the gold-dig-factor (cf. gold-dig-factor in the section Entities, state variables and scales). The idea behind this is that a $\gdf$ of 1 causes behaviour similar to socialist/communist ideas while a $\gdf$ of 0 leads to phenomena similar to capitalism. Capitalism favours accumulation of money, be it in the hands of individuals or corporations \citep{marx}. On the other side of the spectrum socialism favours the equal distribution of money as can be seen in socialist ideas such as progressive taxes \citep{moyes}. We mirror this using the $\gdf$ in the following way: an agent with a high $\gdf$ will, when considering all agents that have proposed him a trade, favour the agent with the least resources over any other agent, no matter how much sympathy he has for any of them. It is obvious that this favours the equal distribution of resources since in a scenario of unequally distributed wealth with all agents having a $\gdf$ of 1 all poor agents will propose trades to (really it's requesting resources from) rich agents which, in turn will accept these trades without consideration. On the other hand a $\gdf$ close to zero will favour the emergence of groups of agents which will exclusively help other agents of the group. Poor agents requesting resources will lose the sympathy of richer agents because requesting resources will lead to a decrease of the sympathy factor while rich agents who don't have to request resources very often will maintain a high sympathy factor.
			
		\item[resource:]
			The one resource of our model can be thought of as being money. Money can take have multiple functions: among others it is a medium of exchange, a measure of value and a store of value \citep{jevons}. In our model the agents need the resource (from now on denoted as money) in order to survive. It acts therefore as a medium of exchange, the agents can, hypothetically, go and buy food and whatever they need in order to survive. At the same time it is a store of value; as we'll see in the next section agents give away money but at the same time they obtain another abstract value, the sympathy of the receiving agent and therefore some form of insurance for the future. The value is not lost but stored in another form. Money also acts as a measure of value, it allows us to quantify the success of individual agents and groups of agents by assigning them and their position a value.
			
		\item[Trading:]
			As we've seen trading in our model is somehow unintuitive: agents can request a trade and then may or may not receive money without really giving something back. No trading actually takes place, it's more giving than trading. The assumption here is a creditor-debitor relationship. A rich agent will give money to a poor agent but both will remember that the trade has happened in the form of the changed sympathy. While the creditor is less likely to give to the same debitor many times, the debitor remembers his debt towards the creditor and is more likely to give him money in case of a request. This resembles the real world notion of credit in so far as the creditor is willing to wait for the debitors situation to improve before claiming back his money. How exactly this creditor-debitor relationship is to be interpreted depends again on the $\gdf$ of both participants: with a low $\gdf$ the actual effect of a trade is accurately described by the above, with a high $\gdf$ not so much since the selection of a trading partner depends less on the sympathy level. This is no contradiction though since a high $\gdf$ mirrors, as described before, the affinity of an agent to distribute wealth. Apart from the obvious distribution of resources the action of trading is an implicit channel of information between two agents. We will investigate this role of trade a bit more below in the section on the logical connection between agents.
			
		\item[Sympathy:]
			The total sympathy level is steadily rising as in each trade the net change of sympathy is +1 in contrast of the somewhat more natural approach of saying, the helper's sympathy towards the helped decreases and the helped agent's sympathy towards it's helper increases. Before explaining this we need to take a step back and explain the term. According to the Cambridge Academic Content Dictionary sympathy can mean two things: "a feeling or expression of understanding and caring for someone else who is suffering or has problems that have caused unhappiness" or "a feeling or expression of support and agreement". We use the latter definition for our model although the two are obviously intertwined in our model sympathy is not a reaction to the state of another agent but rather the state of the relationship to this actor. Among other things the sympathy one feels towards someone else is influenced by the proximity to this person, the similarity of one's state to this persons's state and "one's past and vicarious experiences" \citep{lowenstein}. We express these factors in the way the sympathy factor changes, especially in the fact that the net sympathy is rising. Obviously the proximity between two individuals (thus, also between agents) increases with more interaction and since trade is The interaction in our model it surely should increase the sympathy. On the other hand an agent that has received help has found itself in a situation of need and is thus more likely to being able to relate to other agents being in such a state. 
	\end{description*}
	
	\textbf{\underline{Structure}}:
	\begin{description*}
		\item[Logical Connections:]
			The agents and the edges between them form a graph or a network. When there exists an edge between two agents it means that they "know" each other and that they are able to trade with each other. In this sense the network really captures the meaning of a social network, especially as we assume that all social interaction is trade. As shortly mentioned above the connection between two agents also acts as an implicit channel of information between them: an agent receiving a trade request from one of his neighbours learns something about this agent's situation and, as the situation of an agent mostly depends on his position on the map, also about the surrounding world. A well connected agent is thus better informed than others and is thus more likely to be able to adopt his position (cf: movement) better than others.
			
		\item[Phyisical Connection:]
			Agents which are close to each other are not necessarily logically connected by an edge, nevertheless they share a common piece of information in the shape of their parcel. Close neighbours are likely to be situated on the same parcel, thereby receiving the same amount of resources and having a similar situation. As in the real world the milieu is somehow related to the well-being of its inhabitants. Obviously the physical position of an agent is related to its logical connections. An agent won't, for example, maintain relationships with agents which are very far away for a long time. We ensure this by periodically updating the logical network of an agent to favour closer agents to agents which are further away.
			
		\item[Movement:]
			The last piece of the social structure is the ability of agents to change their position and thus change the structure of model. Although the only thing that changes is the position of the agent on the grid and not its logical place in the network. 
	\end{description*}
	

\subsubsection{Emergence}
	The agents are able to adapt to their environment on their own without influence from the outside. They do this by moving to more beneficial locations on the grid, which they then start to cluster around trying to cover it up as good as possible. This behaviour, as it is not directly modelled, emerges from the small set of rules on which the agents act upon. How good the agents adapt to their environment depends on the situation and on the independent variables.

\subsubsection{Adaptation}
	The agents adapt themselves to the current situation by moving around on the grid. Their basic instinct is not to move to a place where they get less resources than before. This means they explicitly seek to increase their measure of success, namely the amount of resources they possess. This leads to a large-scale migration pattern, in which the agents cluster around places on the grid which yield more benefit.

\subsubsection{Objectives}
	Th explicit objective of our agents is to accumulate as many resources as possible. To achieve this goal, the agents move around the grid and trade with each other, thus distributing the resources more evenly. From this behaviour, the implicit objective of survival emerges. Neither of those objectives are implemented directly, but rather emerge from the rules the agents follow.

\subsubsection{Learning}
	Learning happens collectively: an agent accepting a trade signals to the receiving agent that his position is superior to the one of the other agent. The following movement of the receiving agent is thus an attempt to improve his own position on the grid. This way trading acts as a kind of communication between agents as it does in the real world. Money flowing from urban to rural areas for example often leads to a flow of people from rural to urban areas \citep{davis}

\subsubsection{Prediction}
	 The prediction mechanism built into the agents is the movement reversing. An agent always remembers how much he received in the last iteration and compares this value to how much he has received in the following one. If he has received more in the previous iteration he returns to the old position. This is similar to temporal difference learning in so far as the agents change their prediction of which direction leads to a beneficial location based on the prediction they have made before and on the result this prediction has brought.

\subsubsection{Sensing}
	First and foremost each agent receives feedback in each iteration in the form of how many resources he has received in the parcel he is located on at the current time. This might give him a wrong idea about the true quality of the parcel though since the resources of one parcel are always distributed among all its inhabitants. Very closely related to the previous point we want to emphasize again that the sensation of an agent is enhanced by the network he is connected to: through his neighbours and the amount of resources they have the agent gets a pretty good idea of where the soil could be fruitful.

\subsubsection{Interaction}
	The agents interact with two things: their neighbours, and the grid they are locate on. With their neighbours, they are able to trade and thus to receive or send resources. They also interact spatially, by moving around in a fashion already explained above.

\subsubsection{Stochasticity}
	The age of the agents is modeled with stochasticity: instead of giving each agent an explicit age and fixing the age of death or increasing the chance of death with increasing age we simply set the chance of death for all agents in a certain tick constant to the circle-of-life-factor ($\clf$). This gives an expected age of $\frac{1}{\clf}$ ticks for each agent. The reproduction is modelled analogously, the chance that an agent spontaneously spawns a new agent is equal to $\clf$.

\subsubsection{Collectives}
\subsubsection{Observation}

\subsection{Initialization}
	In the beginning, a fixed number of agents are created and distributed on the grid. While doing this, each parcel counts how many agents are located on itself to be able to later distribute the resource it receives evenly among all agents located on it. Next, neighbours are assigned by connecting nearby agents. The nearer two agent are, the more likely it is that they establish a connection. Each agent receives a starting capital, ranging from 0 ta a fixed point. Since all agents are located on a random place on the grid, many agents are placed in a unfavourable position and will immediately seek to migrate to a better place.
	
	
\section{Results}
We succeeded in building a system which is highly adaptive despite being based solely on very simple rules. Running it with many different distribution functions we learned that the agents will always distribute themselves in a beneficial way. Even under very hard circumstances where the distribution function cuts off all agents from good soil they managed to get back on that good soil due to some random movement. Coming back to our initial question we found that the connectedness of an agent which allows him to propose trades and a certain social preference of its neighbours which will enable money and information flow are crucial for its success. To quantify these findings we've seen that GDF of 1 and WTF of 1 alongside medium connectedness leads to the best results, meaning longest survival and highest adaptivity of the agents. We conclude that in a system with these assumptions very "social" agents (social in the same spirit as described in the section on design principles) are best suited for survival. Preference for itself or for "friends" can only harm the adaptivity of the system as there is a one-to-one correspondence between the number and distribution of an agent's neighbours and this agent's adaptivity and thus its success.  

\section{Discussion}
Despite the abstract nature of this model it nevertheless allows us to draw positive, as well as normative conclusions which we will illustrate in the following.
The cause and process of migration in our model coincides on many levels with theories trying to explain migration. Two ways in which migration is understood are neoclassical economics and relative deprivation theory \citep{jennissen}. Migration from the perspective of neoclassical economics is a problem of demand and supply: Having lower wages in one place and higher wages in another place indicates that the first one has a surplus of labour and a shortage of money while the latter is directly opposite. People moving from the regions with lower wages to the regions with higher wages are, from a neoclassical economical perspective, the logical consequence. We can observe exactly this behaviour in our model: first we check that our agents fulfill the three assumptions of neoclassical economics \citep{weintraub}. Our agents have rational preferences in the form of GDF and WTF and they maximize utility; both of these assumptions have already been broadly explained in the section on design choices. Other than that our agents also decide everything on the basis of full and relevant information, this together with the flow of information in our model has also been discussed in said section. We therefore can conclude that our agents are well-suited for a neoclassical model. Running the model with a realistic distribution function we observe that some time after the random initial distribution of the agents an equilibrium is reached where agents are distributed around the areas of high prosperity; the market has reached an equilibrium.
Relative deprivation theory on the other hand explains migration from another perspective. Instead of viewing the system from a bird's eye perspective it looks at the individuals. Richard Schaefer defines relative deprivation in the following way: "the conscious experience of a negative discrepancy between legitimate expectations and present actualities." \citep{schaefer}. We can easily see that this is what happens in our model if we take a less bird's-eye-like perspective: agents learn that their neighbours fare better than themselves and thus want to improve their own position.
Our model takes overpopulation of certain areas into consideration as resources are always distributed per parcel and all agents on a parcel share the resources it yields. We found that building mechanisms into our model that allow an agent to compare its current position to the position in the last tick vastly improved the longevity of the agents as it reduces overpopulation of certain areas. In the real world we often observe the bad impacts rural flight has on the fleeing people as many find their situation actually worsened compared to before \citep{weeks}. One problem is that big bursts of rural flight lead to an increase in population which can not be dealt with by the existing infrastructure \citep{harris}. This is very similar to the effect in our model before our amelioration. Our model therefore suggests that enabling higher mobility of migrants could lead to better results.

Most of our observations based on running the simulation and observing the behaviour. Another approach would be to actually let the agents learn the best initial configurations. This would provide better ground for answering the question of what kind of society works best in this model. 





% FIGURES AND TABLES

% Figures should be placed in the desired position within the text. Please follow the template below.
% Figure widths can be set using absolute dimensions (e.g., [width = 12cm]) or relative ones (e.g., [width = 0.8/textwidth]). We strongly suggest to use the latter option, as this allows automatic adaptation to different paper widths

%\begin{figure}[!t]
%\centering
%\includegraphics[width=????\textwidth]{????}
%\caption{}
%\label{fig:????}
%\end{figure}

% Tables should be placed in the desired position within the text. Please follow the template below

%\begin{table}[!t]
%	\centering
%	\begin{tabular}{????}
%	\toprule
% 	% first line
%	\midrule
%	% tale body	
%	\bottomrule			
%	\end{tabular}
%	\caption{}
%	\label{tab:????}	
%\end{table}

%%%%%%%%%%%%%%%%%%%%%%%%%%%%%%%%%%%%%%%%%%%%%%
% End of  paragraph numbering. Please leave this untouched
\endparano

%%%%%%%%%%%%%%%%%%%%%%%%%%%%%%%%%%%%%%%%%%%%%%

%%%%%%%%%%%%%%%%%%%%%%%%%%%%%%%%%%%%%%%%%%%%%%

% APPENDICES
% Put your appendices here. Please use the normal sectioning command, e.g.,
% \section{Appendix A: <title of the appendix>}
% \section{Appendix B: <title of the appendix>}
% ...

%%%%%%%%%%%%%%%%%%%%%%%%%%%%%%%%%%%%%%%%%%%%%%

% ENDNOTES. Please uncomment the line below in case of notes.
% \theendnotes

%%%%%%%%%%%%%%%%%%%%%%%%%%%%%%%%%%%%%%%%%%%%%%

% REFERENCES.
% The JASSS bibliographic style file (jasss.bst) is included in the bundle. Please use BibTeX, not BibLaTeX.
% Use natbib commands for references (\citep{}, \citet{}, etc.), not standard LaTeX ones (\cite{}).
% Remember to include the doi and url fields in your bib database. The address field should be included for books.
% Please upload the bib file (not just the bbl one) when submitting.
 
\bibliographystyle{jasss}
\bibliography{paper} % Please set the right name for your bib file

%%%%%%%%%%%%%%%%%%%%%%%%%%%%%%%%%%%%%%%%%%%%%%

 \end{document}